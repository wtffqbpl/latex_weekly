%! Author = yuanjunren
%! Date = 12/11/23

% Preamble
\documentclass[11pt]{article}

\title{My Article}
\author{Yuanjun Ren}
\date{\today}

\bibliographystyle{plain}

% Document
\begin{document}

\maketitle

\begin{abstract}
    This is a paragraph about Pythagorean theorem.
\end{abstract}

\newpage

\tableofcontents
\newpage

\section{Pythagorean theorem in Ascent}
In mathematics, the \emph{Pythagorean theorem} or \emph{Pythagoras' theorem} is a fundamental relation in Euclidean geometry between the three sides of a right triangle. It states that the area of the square whose side is the hypotenuse (the side opposite the right angle) is equal to the sum of the areas of the squares on the other two sides. The theorem can be written as an equation relating the lengths of the sides a, b and the hypotenuse c, sometimes called the Pythagorean equation\footnote{Judith D. Sally; Paul Sally (2007).}:

$$
a^2 + b^2 = c^2
$$

Those two parts have the same shape as the original right triangle, and have the legs of the original triangle as their hypotenuses, and the sum of their areas is that of the original triangle. Because the ratio of the area of a right triangle to the square of its hypotenuse is the same for similar triangles, the relationship between the areas of the three triangles holds for the squares of the sides of the large triangle as well.

\begin{quote}
This proof is essentially the same as the above proof using similar triangles, where some ratios of lengths are replaced by sines.
\end{quote}


\section{Algebraic Proofs}
The theorem can be proved algebraically using four copies of the same triangle arranged symmetrically around a square with side c, as shown in the lower part of the diagram. This results in a larger square, with side $a + b$ and area $(a + b)^2$. The four triangles and the square side c must have the same area as the larger square, $\angle ACB = \pi / 2$.

\begin{equation}
(b + a) ^2 = c^2 + 4ab = c^2 + 2ab
\end{equation}

\subsection{Insert Table with Latex}

\begin{tabular}{|rrr|}
\hline
Rectangle $a$ & Rectangle $b$ & Hypotenuse $c$ \\
\hline
    3         & 4             & 5              \\
    5         & 12            & 13             \\
\hline
\end{tabular}

\bibliography{math}

\end{document}